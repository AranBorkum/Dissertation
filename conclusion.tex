The results given in chapter \ref{chap:analysis} show that one is able to dramatically reduce the background signal acquired in electroweak supersymmetry analysis by placing conscientious cuts on kinematic variables, as well as applying specific requirements to the signal, such as requiring a specific number of final state leptons and requiring only certain combinations of lepton flavours.
It was also seen that by considering off and on $Z$ mass shell events provided different results, with the two different approaches targeting different backgrounds more or less effectively.
Unfortunately, the analysis discussed in this dissertation offers no further motivation for the existence of supersymmetry.



There are two main possibilities for the outcome of a complete analysis; proof of concept, or exclusion of particle mass combinations.
In order to achieve either of these outcomes a further breakdown of the analysis procedure is required.
Firstly, one cannot hope to improve ones signal significance by simply placing exclusion cuts on kinematics.
Other features such as production of jets, the geometry of the production event, and detector responses need to be accounted for, amongst a plethora of other variables acquirable from the events. 
Furthermore, deeper analyses, beyond the scope of an MPhys final year project, apply a binning procedure to variables such as missing transverse energy and the reconstructed invariant mass of the $W$ boson.
In conjunction with this, jet multiplicities become a key delineating factor in the analysis.



It was seen in this analysis that the procedure, as it stands, could undergo some branching to the potential benefit of the outcome.
The main branch off considered is applying different procedures and selection criteria to samples of different mass combinations.
This is motivated by the significant loss in low mass sample yields against the relatively small loss in high mass sample yields.