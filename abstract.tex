Searches for supersymmetry are a forefront region of experimental physics, aiming to uncover new physics beyond the Standard Model.
Discovery of supersymmetry would provide evidence for a whole new family of particles, analogous to those of the Standard Model with variation in their spin.
Additionally, supersymmetry offers a candidate for dark matter, as well as a solution to the hierarchy problem that causes the Higgs mass to be very different from that expected from the theory.
This dissertation discusses an approach to optimising the selection criteria in the search for electroweak supersymmetry; supersymmetry involving the electroweak gauge bosons.
By critically examining the signals and backgrounds produced by Monte Carlo simulations designed to model those occurring in the LHC, a set of selection criteria is defined with the purpose of reducing Standard Model backgrounds and improving the signal significance of the generated sample signals.
In this analysis, we were able to show that one can preserve a good portion of the event signal in the events whilst excluding a vast majority of the background events.
Despite this, the results presented are inconclusive as many more requirements need to be optimised and verified before the validity of the approach can be ascertained. 
This does not rule out the possibility that the results shown could lead to significant results in the analysis, however, further work must be done before a significant result is achievable.