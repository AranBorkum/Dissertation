The backgrounds signals from the analysis discussed in this dissertation can be separated into two classes.
The first of which are irreducible backgrounds, those that mimic the desired signal due to the same signature particles being produced by different interactions.
The second class of backgrounds are the reducible backgrounds.
These are backgrounds that mimic the desired signal through indirect production of the signature particles. 

The SUSY production mechanism considered in this dissertation has the characteristic signature of three leptons, the first and second of which are of the same flavour with opposite charges, and large missing transverse energy.
In this chapter Standard Model interactions that are commonly occurring in high energy interactions that can mimic the desired SUSY signal will be discussed, categorised into reducible or irreducible backgrounds, and methods of attenuation for these backgrounds will be given.

\section{Reducible Backgrounds}
\subsection{$t\bar{t} \rightarrow 2\ell + 2\ \textrm{b-jets} + \textrm{MET}$} \label{subsec:ttbar}
At high enough energies one has the ability to produce a $t\bar{t}$ pair.
The top quark does not live for long enough to hadronise, and so quickly decays into a bottom quark and a $W$ boson.
Subsequently, the bottom quarks will initiate hadronic jets and the $W$ bosons will decay into a lepton and an associated neutrino.
The interaction can thereby be written as:
\begin{align}
t\bar{t} \rightarrow 2\ell  + 2\textrm{b-jets} + E_{T}^{miss}
\end{align}
shown graphically below.
\begin{figure}[htbp] %  figure placement: here, top, bottom, or page
   \centering
   \includegraphics[width=0.3\textwidth]{Pictures/ttbarBKG.png} 
   \caption{$t\bar{t}$ going to two prompt leptons, two b-jets, and missing transverse energy mediated by two neutrinos}
   \label{fig:example}
\end{figure}

\noindent The leptons produced as a result of $W$ decay are referred to as prompt leptons, meaning they are produced as a direct consequence of the initial $t\bar{t}$ event.
Furthermore, it is possible that the two prompt leptons are of the same flavour making them appear to be an SFOS pair.
The issue, however, is that the $b$-jets can indirectly produce leptons, thusly referred to as fakes as they appear to the detector to be produced as a direct consequence of the initial interaction but are, in fact, secondary byproducts. 
For this reason $t\bar{t}$ events are classed at reducible backgrounds.

Due to the fact the the top quark almost exclusively decays to the bottom quark, one can identify this event by the detection of two $b$-jets. 
Furthermore, $b$-jets can be tagged, as discussed in \textbf{\large b jet tagging ref}, and seeing as one expects no $b$-jets in the final state of the SUSY interaction, as a part of the selection criteria one can choose to ignore any event associated with a $b$-jet.

\subsection{$Z+$jets}
The production of a $Z$ boson in conjunction with hadronic jets is a common process in high energy colliders. 
As seen before, the decay of the $Z$ boson can occur through leptonic channels, as follows: 
\begin{align}
Z \rightarrow \ell + \bar{\ell}.
\end{align}
By conservation of charge and lepton number, this lepton pair will always be an SFOS pair. 
This then gives an imitation of the of the SFOS pair expected in the SUSY production mechanism we are interested in.
Similarly to that discussed in \ref{subsec:ttbar}, the jet portion of the event can result in a fake lepton, identified by the detector and assumed to be a part of the primary event.
For this reason $Z +$ jets is a reducible background in ones event signal.

The attenuation of $Z+$jets events is achieved by constraining the invariant mass of the leading two leptons, as discussed in \ref{subsec:MassShells}.
This reduces the background, however, in the analysis discussed in this dissertation $Z+$ jets presents itself as the more prevalent background in the acquired signals. 

\section{Irreducible Backgrounds}
\subsection{Di-boson Events}
In $WZ$ SUSY production, the dominant background comes from the production of $W$ and $Z$ bosons from purely Standard Model processes.
Provided the $W$ and $Z$ both undergo leptonic decay this Standard Model background exactly mimics the signal being looked for.

\begin{figure}[H] %  figure placement: here, top, bottom, or page
   \centering
   \includegraphics[width=0.4\textwidth]{Pictures/WZBackground} 
   \caption{$WZ \rightarrow 3\ell\ +$ MET}
   \label{fig:example}
\end{figure}

\noindent It is more difficult to attenuate this the number of these di-boson event coming though the signal selection process.
Placing constraints on the invariant mass  of the leading and next to leading leptons helps to reduce the number of background event coming though, however, it also reduces the number of potential true signals one has after the signal selection.
Placing further constraints on variables such as missing transverse energy and lepton transverse momentum can also help to minimise the presence of the di-boson background.

Di-bosonic backgrounds dominate the background signal by a vast majority.
Logically this makes sense as the SUSY process we are looking for is very similar to that of the given Standard Model process. 

\subsection{$t\bar{t}V$}
Certain Standard Modal backgrounds that initially come under the category of reducible backgrounds, when combined with an additional gauge boson, can become an irreducible background. 
A good example is with $t\bar{t}V$ backgrounds, particularly when the vector boson, $V$, is a $W$ boson

Under the addition of a $W$ boson, the overall scenario becomes:
\begin{align}
t\bar{t}W \rightarrow 3\ell + 2b\textrm{-jet} + \textrm{MET}
\end{align}

\noindent This is the exact final state being searched for with the addition of two $b$-jets. 
This makes $t\bar{t}V$ background events irreducible backgrounds.
Fortunately, as previously discussed, by tagging the $b$-jets one can infer the presence of an initial top quark.
Therefore, by disregarding events with $b$-jets, $t\bar{t}V$ backgrounds are substantially reduced, making it a more manageable background signal.